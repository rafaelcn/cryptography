% Created 2021-09-15 qua 22:07
% Intended LaTeX compiler: pdflatex
\documentclass[11pt]{article}
\usepackage[utf8]{inputenc}
\usepackage[T1]{fontenc}
\usepackage{graphicx}
\usepackage{grffile}
\usepackage{longtable}
\usepackage{wrapfig}
\usepackage{rotating}
\usepackage[normalem]{ulem}
\usepackage{amsmath}
\usepackage{textcomp}
\usepackage{amssymb}
\usepackage{capt-of}
\usepackage{hyperref}
\usepackage{minted}
\usepackage{indentfirst}
\usepackage{libertine}
\usepackage{tkz-graph}
\usepackage[usenames,dvipsnames]{xcolor}
\usepackage[left=3cm,bottom=3cm,top=2cm,right=2cm]{geometry}
\author{Rafael Campos Nunes\(^1\), Rafael Henrique Nogalha de Lima\(^2\) \(\\\) 19/0098295\(^1\) 19/0036966\(^2\)}
\date{}
\title{O Avançado Padrão de Criptografia (AES)}
\hypersetup{
 pdfauthor={Rafael Campos Nunes\(^1\), Rafael Henrique Nogalha de Lima\(^2\) \(\\\) 19/0098295\(^1\) 19/0036966\(^2\)},
 pdftitle={O Avançado Padrão de Criptografia (AES)},
 pdfkeywords={},
 pdfsubject={},
 pdfcreator={Emacs 26.3 (Org mode 9.1.9)},
 pdflang={Pt_br}}
\begin{document}

\maketitle
\tableofcontents

\(\newpage\)

\section{O AES}
\label{sec:orga84807d}

O algoritmo AES (Advanced Encryption Standard) é uma especificação de
criptografia criado por Rijndael\footnote{Composição de nome para os dois autores originais Vincent Rijmen e Joan Daemen}, sendo uma cifra de bloco de chave
simétrica. Este algoritmo, ao contrário do DES (Data Encryption Standard), não
utiliza uma rede Feistel e é rápido tanto em hardware quanto em software.

A ideia geral do algoritmo é separar a mensagem em blocos e cifrá-las
individualmente. A depender do modo utilizado para cifragem dos blocos
algumas particularidades são adicionadas aos blocos. É importante ressaltar
que esses blocos podem ter tamanhos variados de 128, 192, 256 bits.

A cifragem em blocos do AES implica em uma característica interessante sobre
a execução do algoritmo, ele pode ser paralelizado ou executado em concorrência
já que, em alguns modos, cada bloco é cifrado independentemente do outro.


\subsection{O modo ECB}
\label{sec:org5c67b94}

Esse é o modo mais simples do AES. Basicamente, o \emph{plaintext} é dividido em
blocos, e no caso da implementação do trabalho, ele tem comprimento de 128 bits.
E a cada rodada da criptografia, os dados são preenchidos até que fiquem com o
mesmo comprimento do bloco. Assim, cada bloco será criptografado com a mesma
chave e o mesmo algoritmo. Logo, a vulnerabilidade do algoritmo está nessa
última característica apresentada, pois se criptografarmos o mesmo \emph{plaintext},
obteremos o mesmo \emph{cyphertext}, tendo assim a possibilidade de criptografar e
descriptografar em paralelo.

\subsubsection{Cifração}
\label{sec:org588407b}

O modo ECB é inicializado com /key/(chave) e /round/(número de rodadas).
O primeiro método a ser chamado é o \textbf{\_\_expand\_key}, nele é que ocorre o processo
de expansão da chave, ele tem como entrada uma chave de 16 \emph{bytes}, permitindo
assim a construção de uma matriz linear de 176 \emph{bytes}, o suficiente para
fornecer uma chave de 16 \emph{bytes} para o estágio \textbf{\_\_add\_round\_key} no início da
criptografia e cada uma das 10 rodadas seguintes.

O próximo estágio do modo é o \emph{bytes2matrix}, nele os \emph{bytes} são convertidos em
uma matriz \emph{4x4}. Logo depois o loop, de acordo com o número de rodadas
especificado, é executado sendo que dentro dele há 4 funções. Sendo elas,
\emph{sub\_bytes} em que os 16 \emph{bytes} de entrada são substituídos de acordo com a
constante \emph{rijndael\_box} fornecida pelo algoritmo; resultando assim em uma
matriz \emph{4x4}. A segunda função é \textbf{shift\_rows}, nela cada uma das quatro linhas
da matriz é deslocada para a esquerda, caso haja \emph{leak} dos elementos da matriz
eles serão inseridos no lado direito da linha. A mudança ocorre da seguinte
forma: a primeira linha não é deslocada; a segunda linha é deslocada um (byte)
para a esquerda; a terceira linha é deslocada duas posições para a esquerda; a
quarta linha é deslocada três posições para a esquerda; resultando assim em uma
matriz de 16 \emph{bytes}, mas deslocados em relação um ao outro. A terceira função é
\textbf{mix\_columns}, nela cada coluna de 4 \emph{bytes} é transformada usando uma função
que toma como entrada os 4 \emph{bytes} de uma coluna e gera outros 4 novos \emph{bytes},
substituindo assim a coluna original; resultando em uma nova matriz de 16
\emph{bytes}; essa é a única função que não é utilizada na última rodada. A última
função do laço é a \textbf{add\_round\_key}, nela os 16 \emph{bytes} são considerados 128
\emph{bits} e se for a última rodada, a saída será o texto cifrado, caso contrário,
o resultado de 128 \emph{bits} são interpretados como 16 \emph{bytes} e é novamente
iniciada uma nova rodada. Após o laço, as funções de \textbf{sub\_bytes}, \textbf{shift\_rows} e
\textbf{add\_round\_key} são executadas novamente e por fim a função de \emph{encrypt} no
modo ECB retorna um conjunto de bytes da imagem cifrada.

\subsubsection{Decifração}
\label{sec:orgce52941}

O processo de decifração no modo ECB é semelhante ao processo de cifração,
porém, é realizado na ordem inversa. Cada rodada consiste em 4 processos
conduzidos na ordem inversa: \textbf{add\_round\_key}, \textbf{inv\_mix\_columns},
\textbf{inv\_shift\_rows}, \textbf{inv\_sub\_bytes}. O retorno da função é a imagem em \emph{bytes}
decriptada.

\subsection{O modo CTR}
\label{sec:org97f4029}

Esse modo é um dos mais complexos e completos do AES. Nele é utilizado um
contador que tem o mesmo tamanho do bloco usado. No CTR, a operação XOR com o
\emph{plaintext} é realizada no bloco de saída do criptografador. Todos os blocos de
criptografia usam a mesma chave de criptografia. Portanto, para garantir a maior
segurança é necessário que a chave seja alterada a cada \(2^\frac{n}{2}\) blocos de
criptografia.

\subsubsection{Cifração}
\label{sec:org653243a}

O modo CTR possui algumas funções em comum com o modo ECB, assim como com os
demais modos do modelo AES. Por isso, comentaremos somente as diferentes funções
do modo ECB. Esse modo é inicializado com um parâmetro
(\emph{iv} - vetor de inicialização) a mais em relação ao modo ECB, esse vetor é o
bloco “falso” para o processo de cifragem nesse modo que gera mais aleatoriedade
na função. Além disso, o incremento é realizado pela função \textbf{text\_\_inc\_bytes}. Uma
diferença importante entre o modo ECB e CTR é o uso de XOR, nno modo CTR, em
cada bloco gerado com o auxílio do vetor de inicialização para gerar mais
aleatoriedade na cifragem.  Ao final é retornado um vetor de \emph{bytes} da imagem
encriptada.

\subsubsection{Decifração}
\label{sec:orgb2ea0cd}

O processo de decifração no modo CTR é semelhante ao processo de cifração,
porém, é realizado na ordem inversa. O retorno da função é a imagem em \emph{bytes}
decriptada.

\section{Arquitetura do Projeto}
\label{sec:org39f0397}

O projeto foi escrito na linguagem Python utilizando-se de módulos para
organização do código. O AES, neste projeto, é um módulo contendo dois arquivos
que implementam o modo ECB e CTR. Além disso, há abstrações para manipular
imagens no formato bitmap para que seja possível visualizar o resultado da
cifragem e decifragem.

Para implementar a o registro da imagem em modo bitmap corretamente, é
necessário extrair o cabeçalho e o corpo da imagem, cifrar o corpo e inserir o
cabeçalho no corpo cifrado. Deste modo, a imagem é corretamente interpretada.

\(\newpage\)

\section{Resultados}
\label{sec:org61737e1}

Para obtenção dos resultados foi necessário utilizar cinco ciclos de
criptografia e o comando para inicialização do algoritmo pode ser visto abaixo.

\begin{minted}[frame=lines,linenos=true]{shell}
python3 src/main.py -i assets/medium.bmp -k "r@f@el" -c 5 -m ecb
python3 src/main.py -i assets/medium.bmp -k "r@f@el" -c 5 -m ctr
\end{minted}

Ao executar o algoritmo no modo ECB, observou-se os cinco ciclos de cifragem e
decifragem da imagem. As imagens estão organizadas abaixo, sendo incluído
somente 4 para simplificação.

\begin{figure}[ht]
  \center
  \label{ fig7}
  \begin{minipage}[b]{0.4\linewidth}
    \centering
    \includegraphics[width=.4\linewidth]{assets/documentation/medium.bmp.dec-1.ecb.png}
    \caption{Ciclo 1 de AES (enc)}
    \vspace{4ex}
  \end{minipage}%%
  \begin{minipage}[b]{0.4\linewidth}
    \centering
    \includegraphics[width=.4\linewidth]{assets/documentation/medium.bmp.dec-0.ecb.png}
    \caption{Ciclo 1 de AES (dec)}
    \vspace{4ex}
  \end{minipage}
  \begin{minipage}[b]{0.4\linewidth}
    \centering
    \includegraphics[width=.4\linewidth]{assets/documentation/medium.bmp.enc-1.ecb.png}
    \caption{Ciclo 2 de AES (enc)}
    \vspace{4ex}
  \end{minipage}%%
  \begin{minipage}[b]{0.4\linewidth}
    \centering
    \includegraphics[width=.4\linewidth]{assets/documentation/medium.bmp.enc-0.ecb.png}
    \caption{Ciclo 2 de AES (dec)}
    \vspace{4ex}
  \end{minipage}
\end{figure}

As imagens mosram, claramente, como os ciclos de cifragem alteram a imagem. Isto
é, a imagem decifrada do ciclo dois (imagem 4) corresponde exatamente com a
figura cifrada do ciclo 1 (imagem 1).

A execução do modo CTR foi realizada de modo análogo, com uma ressalva, foi
possível indicar que a entropia da imagem é maior do que o modo ECB.

\begin{figure}[ht]
  \center
  \label{ fig7}
  \begin{minipage}[b]{0.4\linewidth}
    \centering
    \includegraphics[width=.4\linewidth]{assets/documentation/medium.enc-0.ctr.png}
    \caption{Imagem cifrada com CTR}
    \vspace{4ex}
  \end{minipage}%%
  \begin{minipage}[b]{0.4\linewidth}
    \centering
    \includegraphics[width=.4\linewidth]{assets/documentation/medium.dec-0.ctr.png}
    \caption{Imagem decifrada com CTR}
    \vspace{4ex}
  \end{minipage}
\end{figure}

\section{Problemas}
\label{sec:orgdd05abc}

O fator limitante na execução do algoritmo em grandes imagens foi a escolha da
linguagem além de que não foi implementado qualquer tipo de mecanismo de
concorrência ou paralelismo afim de aumentar a velocidade do processamento em
computadores com mais núcleos.

O material na internet divergia bastante e foi necessário fazer um grande filtro
e vários testes para verificar o que funcionava ou não, com respeito a
manipulação de imagens.

A manipulação de imagens no formato png ou jpeg/jpg é mais complexa pois dados
da imagem permeiam os dados de cor, dificultando a edição dos dados tal qual é
feito de maneira simples utilizando bitmaps.

\section{Configuração de Ambiente}
\label{sec:orgfad789a}

O ambiente utilizado para construção e teste do trabalho é o GNU/Linux, com o
python na versão 3.6.9. No Windows o python3 é instalado com o nome python.
certifique-se de que está utilizando a versão correta com \texttt{python -{}-version}.

Se for desejo do corretor, o trabalho também contém testes que podem ser
executados afim de demonstrar a corretude do algoritmo. Para isso basta instalar
a única dependência necessária (Crypto) através do pip com o seguinte comando:

\begin{minted}[frame=lines,linenos=true]{shell}
$ pip install -r requirements.txt
\end{minted}

Após instalar a dependência, execute o script \texttt{src/(ctr|ecb)\_test.py}. O CTR e o
ECB vão falhar de imediato pois a implementação desses na biblioteca Crypto é
diferente da realizada por nós. Isso porquê existem algumas diferenças nas APIs
de construção da cifra.

A título de exemplo o algoritmo de CTR é inicializado, com a biblioteca Crypto,
utilizando uma função interna de contagem.

\subsection{Execução e testes do programa}
\label{sec:org9ed9a08}

O programa na sua configuração atual suporta diversos tipos de parâmetros, todos
listados e documentos ao executar o comando \texttt{python3 src/main.py -{}-help}.
Exemplos de utilização são listados abaixo.

\begin{minted}[frame=lines,linenos=true]{shell}
$ python3 src/main.py -i assets/github_profile.png -k "sua chave"
$ python3 src/main.py -i assets/github_profile.png -k "sua chave"
  -m ctr
$ python3 src/main.py -i assets/github_profile.png -k "sua chave"
  -m ctr -c 5
\end{minted}

Dois dos argumentos que podem confundir o usuário é o \texttt{-c} e o \texttt{-r}. O primeiro
diz respeito ao número de ciclos de cifração e decifração que será aplicado ao
arquivo e o segundo é o número de rodadas que o algoritmo irá executar para
cifrar o arquivo.

Por fim, a execução do programa resulta em arquivos com sufixos \textbf{dec} ou \textbf{enc},
juntamente ao índice do ciclo do algoritmo na pasta em que o programa python foi
iniciado.

\(\newpage\)

\section{Referências}
\label{sec:orgb8d8158}

\begin{enumerate}
\item \url{https://csrc.nist.gov/csrc/media/projects/cryptographic-standards-and-guidelines/documents/aes-development/rijndael-ammended.pdf} \(\\\)
\label{sec:org6225edc}
\item \url{https://pillow.readthedocs.io/en/stable/reference/Image.html} \(\\\)
\label{sec:org95cd7d8}
\item \url{https://xilinx.github.io/Vitis\_Libraries/security/2020.1/guide\_L1/internals/ecb.html} \(\\\)
\label{sec:orgf424d13}
\item \url{https://www.intechopen.com/chapters/67728} \(\\\)
\label{sec:org5153775}
\item \url{https://www.gta.ufrj.br/grad/10\_1/aes/index\_files/Page588.htm} \(\\\)
\label{sec:org443da9b}
\item \url{https://asecuritysite.com/subjects/chapter88} \(\\\)
\label{sec:org6326958}
\item \url{https://www.intechopen.com/chapters/67728} \(\\\)
\label{sec:orgc075500}
\end{enumerate}
\end{document}
