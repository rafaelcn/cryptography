% Created 2021-09-14 ter 17:29
% Intended LaTeX compiler: pdflatex
\documentclass[11pt]{article}
\usepackage[utf8]{inputenc}
\usepackage[T1]{fontenc}
\usepackage{graphicx}
\usepackage{grffile}
\usepackage{longtable}
\usepackage{wrapfig}
\usepackage{rotating}
\usepackage[normalem]{ulem}
\usepackage{amsmath}
\usepackage{textcomp}
\usepackage{amssymb}
\usepackage{capt-of}
\usepackage{hyperref}
\usepackage{minted}
\usepackage{indentfirst}
\usepackage{libertine}
\usepackage{tkz-graph}
\usepackage[usenames,dvipsnames]{xcolor}
\usepackage[left=3cm,bottom=3cm,top=2cm,right=2cm]{geometry}
\author{Rafael Campos Nunes\(^1\), Rafael Henrique Nogalha de Lima\(^2\) \(\\\) 19/0098295\(^1\) 19/0036966\(^2\)}
\date{}
\title{O Avançado Padrão de Criptografia (AES)}
\hypersetup{
 pdfauthor={Rafael Campos Nunes\(^1\), Rafael Henrique Nogalha de Lima\(^2\) \(\\\) 19/0098295\(^1\) 19/0036966\(^2\)},
 pdftitle={O Avançado Padrão de Criptografia (AES)},
 pdfkeywords={},
 pdfsubject={},
 pdfcreator={Emacs 26.3 (Org mode 9.1.9)},
 pdflang={English}}
\begin{document}

\maketitle
\tableofcontents

\(\newpage\)

\section{Introdução}
\label{sec:org114eaa4}

\section{O AES}
\label{sec:org4a4cb4b}

O algoritmo AES é uma evolução do DES - sendo este utilizador de estruturas de
Feistel -

\subsection{asd}
\label{sec:orgba26822}
\subsubsection{asd}
\label{sec:org8921222}
\begin{enumerate}
\item asdasd
\label{sec:orgdb27f56}
\begin{enumerate}
\item asd asdasd
\label{sec:org70a4021}
\end{enumerate}
\end{enumerate}

\section{Configuração de Ambiente}
\label{sec:org9f683aa}

O ambiente utilizado para construção e teste do trabalho é o GNU/Linux, com o
python na versão 3.6.9.
\end{document}
