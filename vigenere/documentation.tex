% Created 2021-08-13 sex 18:22
% Intended LaTeX compiler: pdflatex
\documentclass[11pt]{article}
\usepackage[utf8]{inputenc}
\usepackage[T1]{fontenc}
\usepackage{graphicx}
\usepackage{grffile}
\usepackage{longtable}
\usepackage{wrapfig}
\usepackage{rotating}
\usepackage[normalem]{ulem}
\usepackage{amsmath}
\usepackage{textcomp}
\usepackage{amssymb}
\usepackage{capt-of}
\usepackage{hyperref}
\usepackage{minted}
\usepackage{indentfirst}
\usepackage{libertine}
\usepackage{tkz-graph}
\usepackage[usenames,dvipsnames]{xcolor}
\usepackage[left=3cm,bottom=3cm,top=2cm,right=2cm]{geometry}
\author{Rafael Campos Nunes \(\\\) 19/0098295}
\date{}
\title{Cifra de Vigenère}
\hypersetup{
 pdfauthor={Rafael Campos Nunes \(\\\) 19/0098295},
 pdftitle={Cifra de Vigenère},
 pdfkeywords={},
 pdfsubject={},
 pdfcreator={Emacs 26.3 (Org mode 9.1.9)},
 pdflang={English}}
\begin{document}

\maketitle
\tableofcontents

\(\newpage\)

\section{A Cifra de Vigenère}
\label{sec:org17a65a6}

A cifra de Vigenère é caracterizada como uma cifra de fluxo simétrica pois atua
em cada caractere (\emph{byte}) que passa pelo algoritmo, transformando texto em
criptograma que, por fim, é retornado. Ele é considerado simétrico pois utiliza
a mesma chave para as etapas de cifragem e decifragem.

Uma característica importante da cifra é que ela é polialfabética, o que reduz
ataques relacionados à frequência de caracteres do criptograma e, embora isso
seja possível, a análise do criptograma através da frequência de palavras é um
pouco mais complicada pois envolve mecanismos que não são conhecidos \emph{a priori},
tal como o tamanho da chave utilizada para cifrar o texto.

\subsection{Cifração}
\label{sec:org45c9033}

A cifra do algoritmo é um processo similar ao utilizado na cifra de Caesar, com
a diferença de utilizar um polialfabeto para transformar o texto puro em
criptograma.

O processo de cifragem de uma mensagem \(M\) com uma chave \(K\) de tamanho \(n\) é
denotado pela equação abaixo para um alfabeto definido na tabela ASCII.

\begin{equation}
\label{eq:1}
C_i = M_i + K_{i \mod len(K)}
\end{equation}

A soma definida na equação \(\ref{eq:1}\) é uma soma aritmética da posição da
letra \(M_i\) na tabela ASCII com a letra pertencente a chave \(K_{i \mod len(K)}\)
também na tabela ASCII. O resultado é um criptograma \(C\) com tamanho idêntico à
mensagem \(M\).

\subsection{Decifração}
\label{sec:org5f9e193}

O processo de decifragem é análogo ao de cifragem, com a diferença na operação
que, ao invés de somar é realizado a operação de subtração sobre a mensagem,
como denota a equação abaixo

\begin{equation}
\label{eq:1}
M_i = C_i - K_{i \mod len(K)}
\end{equation}

\section{Utilização da ferramenta}
\label{sec:org1777b73}

A ferramenta pode ser utilizada de acordo com as opções definidas em --help.

\begin{minted}[frame=lines,linenos=true]{shell}
$ ./vigenere.exe --help
\end{minted}

\section{Limitações da ferramenta}
\label{sec:orgc802874}

Apesar do programa ter a função de realizar a criptoanálise sobre um texto,
é importante ressaltar que essa configuração só funcionará para chaves de até
100 caracteres. Esse número não é por acaso, foi limitado dessa maneira para
facilitar a análise de grandes corpos de texto em que se supõe que o número
de coincidências do criptograma com ele mesmo em diferentes posições não terá
uma distância maior do que 100 caracteres.
\end{document}
