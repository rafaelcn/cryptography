% Created 2021-10-25 seg 23:54
\documentclass[11pt]{article}
\usepackage[latin1]{inputenc}
\usepackage[T1]{fontenc}
\usepackage{fixltx2e}
\usepackage{graphicx}
\usepackage{longtable}
\usepackage{float}
\usepackage{wrapfig}
\usepackage{rotating}
\usepackage[normalem]{ulem}
\usepackage{amsmath}
\usepackage{textcomp}
\usepackage{marvosym}
\usepackage{wasysym}
\usepackage{amssymb}
\usepackage{hyperref}
\tolerance=1000
\usepackage{minted}
\usepackage{indentfirst}
\usepackage{libertine}
\usepackage{tkz-graph}
\usepackage[usenames,dvipsnames]{xcolor}
\usepackage[left=3cm,bottom=3cm,top=2cm,right=2cm]{geometry}
\author{Rafael Campos Nunes$^1$, Rafael Henrique Nogalha de Lima$^2$ $\\$ 19/0098295$^1$ 19/0036966$^2$}
\date{}
\title{Criptografia Assim�trica (RSA)}
\hypersetup{
  pdfkeywords={},
  pdfsubject={},
  pdfcreator={Emacs 25.3.1 (Org mode 8.2.10)}}
\begin{document}

\maketitle
\tableofcontents


\section{Introdu��o}
\label{sec-1}



\section{Arquitetura}
\label{sec-2}

O projeto foi escrito na linguagem Python utilizando-se de m�dulos para
organiza��o do c�digo. A pasta RSA, neste projeto, � um m�dulo contendo
que implementam . Al�m disso,


\section{Problemas}
\label{sec-3}



\section{Ambiente}
\label{sec-4}

O ambiente utilizado para constru��o e teste do trabalho � o GNU/Linux, com o
python na vers�o 3.6.9. No Windows o python3 � instalado com o nome python.
certifique-se de que est� utilizando a vers�o correta com \verb~python --version~.
% Emacs 25.3.1 (Org mode 8.2.10)
\end{document}
